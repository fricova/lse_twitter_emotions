\begin{abstract}
\phantomsection\addcontentsline{toc}{chapter}{Abstract}
At the beginning of the global Covid-19 outbreak in February and March 2020, the Standard \& Poor 500 (S\&P500) index suffered one of its most sizable historical crashes. Then, in the months that followed, a volatile price rise brought the S\&P500 index back to, and even beyond, its pre-pandemic price levels. The purpose of this paper is to investigate whether the stock market developments between February and October 2020 could be predicted through public mood expressed on social media. To estimate the public mood during the Covid-19 health emergency, we calculate the daily prevalence of fear, joy, anticipation and trust expressions, as well as the overall sentiment index of Covid-19 themed posts on Twitter. We then test the predictive power of each aggregated emotion series and the sentiment index for S\&P500 prices by using a linear Granger causality model and a fully connected neural network. Our results indicate that the accuracy of technical S\&P500 forecasts can be substantially improved with the inclusion of multiple aggregate emotion metrics. Twitter fear, in particular, consistently linearly predicts S\&P500 prices for up to six trading days ahead. Also, including emotion metrics in our fully connected neural network enhances out-of-sample prediction. The unidimensional Twitter sentiment index, on the other hand, brings little forecasting value. Our findings might be of interest to equity index investors and, furthermore, they offer theoretical support to the behavioural finance theory linking public mood with investor behaviour. 

\end{abstract}